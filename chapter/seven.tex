\chapter{总结与展望}
\section{论文工作总结}
陕西某油田企业作为气井行业的重要参与者,随着气井数量的增多及老井管理难度的提升,正面临着日益严峻的数字化建设挑战。企业历来依赖传统经验进行生产管理和决策制定,但随着技术的发展和市场的变化,此种做法已不再适应当前的行业发展需求。
此外,由于早期对数据管理缺乏足够的重视,导致大量宝贵的数据信息散布在各个不同的平台上,造成数据孤岛现象,同时很多员工手握大量线下数据,这不仅加大了信息整合的难度,也影响了企业的决策效率和精确度。

在这样的背景下,本文提出了一个创新的解决方案——气井智能管控系统。该系统旨在通过引入机器学习、大数据分析等先进技术,实现对气井全生命周期的智能管理,以此来提高气井生产管理和决策制定的质量与效率。

本文的主要研究工作可以概括为以下几个方面:

(1)鉴于不同气井显示出不同的生产特性,本文提出了将营销分析中的RFM模型应用于气井分类的创新方法。RFM模型原本用于分析客户价值和进行客户细分,但本文巧妙地将其运用于气井分类,通过结合RFM模型和核密度估计技术,成功开发出一种新型的气井分类算法。
相较于传统的聚类算法和时间聚类算法,这种新方法在分类效果上表现更为出色,且结果更加稳定,可为企业提供更加准确的数据支持。

(2)本文针对企业在决策制定上的盲目性和经验依赖问题,需要对气井产量进行预测的需求,提出了一种基于transformer算法的产气量预测模型。该模型能够综合利用混合输入,包括气井号这类静态信息、每日产量这类过去已知未来不可知信息和气井已生产天数等过去未来都已知的历史数据,
采用GRU来学习局部特征,并利用自注意力机制学习不同时间步之间的长期依赖关系。通过这种方式,模型能够更准确地预测未来的产气量,为企业提供更科学、更精确的数据支持,以优化生产管理和决策过程。同时通过实验证明先对气井分类再进行产气量预测可以提升预测准确率。

(3)在进一步的智能分析过程中,该陕西油田企业出现了的软件应用分散和数据孤岛问题,针对该问题,本文进行了详细的需求分析,在此基础上设计了包括数据连接、数据管理和智能分析三大模块的气井智能管控系统。该系统不仅能有效整合分散在不同平台的数据资源,解决信息孤岛问题,
还能通过其智能分析模块提供
气井分类、产气量预测和开关井推荐等功能,极大地提升了数据的有效利用率和企业的决策质量。
\section{后续工作展望}
尽管本文提出的气井智能管控系统在一定程度上缓解了企业当前面临的若干问题,但企业在智能分析领域仍有广泛的探索和发展空间。具体而言,企业的智能分析工作可分为以下几个重点领域:

(1)目前系统中的开关井推荐机制主要基于管网模型,尚未充分考虑到开关井与气井积液之间的密切相关性。因此,下一步工作需着重于气井产液量的精确预测。企业应开发一种综合算法,该算法不仅要考虑产液量,还要将其他相关因素纳入考量,以制定出更加符合企业实际需求的开关井策略。

(2)在通过数据连接获取了气井储层的关键参数(例如储层压力和温度)以及一些技术操作数据(例如钻井和完井情况)之后,可以进一步利用这些数据来构建更加精确和细致的地质及储层模型。这些模型将有助于深入理解储层的结构和流体的分布情况。
同时,借助三维可视化技术,对储层的孔隙度、渗透率和油气饱和度进行系统的详细分析,这不仅能为钻井位置的选择提供科学依据,也能显著提高开采效率。

(3)企业应将储层和生产数据应用于预测性维护的实践中,从而及时识别出可能需要干预或修井的气井。通过深入的智能分析,可以确定最佳的干预时机和方法,这不仅能有效降低停机时间,还能大幅减少维护成本。