\chapter{绪论}
本章主要阐述了气井智能管控系统的研究背景及意义,对国内外油气数字化现状进行了调研。进一步解释了对气井进行分类、产量预测和开关井推荐的必要性。最后对论文主要研究内容和组织结构进行了简要说明。
\section{研究背景和意义}
天然气是世界上广泛使用的能源。2022年,世界天然气消费量3.94万亿立方米,全球油气勘探开发投资支出4934亿美元,其中我国全国天然气
消费量3646亿立方米,进口天然气1503亿立方米\cite{chinaGasGOv}。在我国的能源资源中有举足轻重的地位。

随着计算机技术的快速发展和软件技术的高速迭代,利用计算机技术来应用于传统行业已经十分有必要。
尽管我国原材料工业数字化转型不断走向纵深,但仍面临计算机技术应用于产业不够深入等问题\cite{workplanForPetrochemical}。油气行业面对气井智能分析的需求,迫切的需要利用计算机技术进
行数字化的建设和发展。本文将使用计算机方法来对气井进行分类、对气井产气量预测、根据预测结果进行开关井推荐,并以此为基础设计一个包含数据管理和智能分析的气井智能管控系统。

目前,已经有相当多的学者开展了气井分类方
面的研究,但当前绝大多数的分类方法需要基于
现场管理认识,给出指定的分类界限,对管理经验要
求较高的同时,还具有相当的主观性\cite{SYZC202104015}。同时,苏里格气田是典型的三低(低渗透、低压力、低丰度)气田,不仅单井产量低,产量递减快,压力下降快,且积液严重,大多数为间歇性气井\cite{KTSY202306014}。
对于该气田而言,不同井的特征随时间变化的差异很大,相对于基于传统特征的分类方法,采用时间序列聚类分析方法可以深入分析气井随时间变化的产气量,允许用户精准地识别出不同生产条件下气井的行为模式。时间序列聚类能够基于实际的生产动态,而不仅仅是静态和动态特征,提供更个性化、适应性更强的气井预测算法,从而可以根据预测结果来提供开关井策略。这有助于最大化生产效率,减少液体积累带来的问题,优化生产过程,从而在不同的生产阶段维持气井的最佳运行状态,提高整体的生产效能。

气井产能预测是气田开采过程中最基本的任务之一。常用的传统的预测气井产量的方式有两种:预测气井的绝对无阻流量和通过递减曲线分析来预测产量。气井的无阻流量指的是在气井井底流压降低到0.1MPa时,气井的产气量。实测无
阻流量指的是将气井彻底敞开,当井底流动压力和大气压保持相同时,测量产气量, 此即为实测无阻流量。 该方法被认为是不合理的 , 原因在于气管尺寸会影响到结果的准确
性。而且操作过程中形成的气体会有严重的消耗,在水的锥进以及砂子颗粒的磨损作用下气井会受到影响。递减曲线分析方法依赖于简化假设和历史生产数据,这可能不完全反映复杂的地质和工程条件,特别是在非常规气藏中。此外,该方法忽略了井筒和地层之间的非均质性,需要长期的生产历史来形成准确的分析,且未考虑经济因素,从而可能导致对未来产量的预测不够准确或适用。

鉴于传统预测方法有诸多弊端,很多学者采用机器学习如时间序列预测等方法来预测气井产量,目前油气产量时间序列预测的研究大多是同步时间序列预测,问题描述为:已知属性(Source)和预测目标(Target)在每一步相对应,且已知属性时间序列长度和预测目标时间序列长度相等时计算产量。同步时间序列预测实际上是对未测量数据的计算,因此,能够应用的场景有限。
本文通过将静态变量,现在已知的未来变量和过去的动态变量融合到同一个网络中,可以进行多时间尺度的预测。

在对气井进行预测后,企业可以根据预测结果来进行开关井推荐,使得生产决策更加数据驱动,确保资源的有效利用。气井的产量通常受到管网两端压力差的限制。当管网两端的压力差较大时,气井的产量会增加;而当压力差较小时,气井的产量会减少。相应地,开关井操作会改变管网两端的压力差。一般情况下,开井会导致管网两端的压力差减小,而关井则会导致压力差增大。此外,当管网两端的压力差发生变化时,会对气井的稳定性产生影响。特别是在压力差较小时,气井的稳定性容易受到影响。因此,在开关井控制中,需要综合考虑管网两端的压力差和气井的稳定性因素,制定最优的开关井策略,以确保气井的生产稳定性。
本文采用果蝇优化算法,根据用户期望产量,基于管网模型计算理论值与计算值的差值,来作为开关井选择的标准。

同时,陕西某油田企业不同部门的数据杂乱无章,数据格式不统一,内容不共享,用户往往需要维护多个平台的账号信息,学习每个平台的使用方式,使用Excel等格式文件在各平台间导入导出数据,导致管理十分混乱。随着企业战略发生变化,需要高度整合自身资源,避免未来的智能分析工作陷入瓶颈,系统设计了数据连接和数据管理功能。

受企业委托,本文提出了基于机器学习的气井智能管控系统,通过现有理论完成气井分类、产气量预测和开关井工作,并根据企业需求设计了气井智能管控系统。系统不仅包括气井分类、产气量预测和开关等推荐等智能分析的内容,还根据企业数据杂乱管理混乱的痛点
设计了数据连接、数据管理、可视化展示和用户权限管理等功能。本文系统已完成一期交付,正处于线上试用阶段,目前运行状况良好。

\section{国内外研究现状}
\subsection{时间序列聚类算法}
时间序列聚类可以很好地用于趋势分析和预测,在股票市场分析、气象预报等领域已经有了广泛应用。时间序列聚类与常规的数据聚类方法存在显著区别,主要体现在处理数据的方式上。其中一种方法是基于形状的聚类,它通过调整时间序列在时间轴上的延展和收缩来匹配两个序列的形状。C.Ratanamahatana等于2006年提出了一种混合聚类算法,这种算法先根据动态时间弯曲(Dynamic Time Warping, DTW)的相似性度量方法将时间序列分成不同的小组,然后利用K-Means和K-Medoids算法进行聚类。另一种方法是基于特征的聚类,其中V. Hautamaki和其他学者在2008年介绍了一种方法,该方法首先将每个时间序列转换成等长的特征向量,然后使用欧几里得距离来实现聚类。这些方法都是为了更好地解决时间序列数据的特殊性,以揭示数据中隐藏的模式和结构。
时间序列聚类的关键在于如何准确测量不同时间序列之间的相似性。这个概念最初由R.Agrawal及其同事在1993年提出,并迅速成为数据挖掘领域的重要问题。在寻找时间序列之间相似形状的传统方法中,欧氏距离和动态时间规整(DTW)距离是常用的指标。近年来,研究人员探索了新的方法来改进这一过程。例如,S. Chu在2002年基于DTW开发了一种名为迭代深化动态时间规整(IDDTW)的方法。此外,为了提高时序聚类的通用性、精确度和效率,John Paparrizos与Luis Gravano在2016年引入了一种新的基于形状的距离度量(SBD),这种方法不受时间伸缩和平移的影响,增强了其适用性和扩展性。
\subsection{时间序列预测算法}
在时间序列分析的历史进程中,自回归模型(AR),首次提出于1927年,为统计学在时间序列预测领域的应用奠定了基础。随后,自回归积分移动平均(ARIMA)模型的发展,标志着基于统计学的预测方法进入了一个新阶段,ARIMA模型因其综合了自回归、积分和移动平均的特性而成为后续研究的核心。此后的模型大多在此模型的基础上进行改进。
然而这些模型的建立和预测精度在很大程度上依赖于参数选择,而不当的参数设定可能导致预测效果不佳,尤其是在面对复杂和动态变化的数据时。因此,尽管统计学模型在时间序列分析领域内有着长久的应用历史和理论基础,但在实际应用中仍需要细致的参数调整和模型选择。
机器学习在处理时间序列数据时表现的十分优秀。支持向量机是一种二分类模型,但可以通过多个支持向量机进行回归预测。在时间序列预测中,支持向量机常用的核函数包括线性核函数、多项式核函数和径向基核函数。
随机森林是一种集成学习算法,其基本单元为决策树。在时间序列预测中,随机森林可以对多个决策树进行集成,从而提高预测的准确性。集成学习是一种将多个弱分类器组合成强分类器的机器学习算法。在时间序列预测中,集成学习可以通过组合多个基模型来提高预测的准确性。常用的集成学习[11]算法包括Adaboost、Bagging和随机森林等。其中LightGBM是一个基于Boosting的集成树模型,由微软亚洲研究院开发并开源,其通过构建梯度提升决策树来预测时间序列。该框架特别适合处理大规模数据集,因为它采用了直方图优化算法和叶子优先的树增长策略,显著降低了内存消耗和提升了计算速度。LightGBM还支持分类特征和连续特征,使其能够灵活处理各种类型的时间序列数据。此外,通过应用自动特征选择和过拟合防止策略,LightGBM能够生成准确且鲁棒的时间序列预测模型。
此外,还有一系列神经网络算法也应用于时间序列预测领域。循环神经网络(RNN)是1990年由Elman首次提出,用于处理序列数据。RNN通过在模型中设置隐藏层来保存过去的状态,并将其作为后续输入的一部分,以此来捕获时序信息。然而,RNN在处理长期依赖问题时表现不佳,因此发展出了长短期记忆网络(LSTM)和门控循环单元(GRU)等改进型网络。LSTM通过引入控制门机制来改善长时间依赖的捕获,而GRU则简化了LSTM结构,降低了模型复杂度。此外,为了解决RNN只能利用单向数据的限制,双向循环网络(Bi-LSTM)被提出,使模型能同时利用过去和未来的信息。自2017年由Vaswani等人提出Transformer模型以来,其已成为深度学习领域的革命性创新。该模型通过引入自注意力机制,有效改进了对长期依赖性的学习,显著提升了自然语言处理等领域的性能。尽管其在长序列预测上存在挑战,如二次时间复杂度,但其并行计算能力和高效的自注意力机制使其在多个领域得到广泛应用,展现出优越的性能。此外,最新研究也表明,注意力机制在时间序列预测方面相比传统模型展现出更好的效果。如TFT(Temporal Fusion Transformers),其结合了LSTM和自注意力机制的优势,能够有效地捕获时间序列中的长期和短期依赖性。TFT 的核心是一个多头自注意力机制,它允许模型同时关注过去、现在和未来的时间点,提高了预测的准确性和解释性。此外,TFT 通过其可变选择网络 (Variable Selection Networks) 部分,能够自动识别并选择对预测任务最相关的输入特征,这不仅提高了模型的效率,也增加了模型结果的可解释性。
\subsection{机器学习在油气产量的应用}
前文可知,有多种机器学习算法应用于时间序列预测领域,其在油气田产量预测领域的应用也十分广泛。
Hui团队考虑了地质和操作变量,测试了线性回归、人工神经网络、梯度提升树以及极端树四种不同的模型,最终发现极端树模型在预测页岩气产量方面最为有效,并据此提出了提高产量的建议。
Liao团队开发了一个基于数据挖掘的模型用于致密油产量预测,通过敏感性分析确认了关键变量,并确定随机森林为最适合的模型。
Niu等人应用了随机森林、支持向量机、K最邻近和梯度提升树四种机器学习方法来预测油田的最终采收率,其中支持向量机的表现最优,并指出早期生产数据对预测结果有重要影响。
Li等人的研究聚焦于使用机器学习方法优化压裂产能预测,经过变量筛选后,发现随机森林模型提供了最佳预测结果,并基于此提出了施工优化建议。
Morozov及其团队执行了一个全面的累积油产量预测研究,涵盖了从数据处理到模型比较的整个过程。他们测试了多种机器学习技术,包括支持向量机(SVM)、K最近邻(KNN)、人工神经网络以及多种基于树的算法,如决策树、随机森林、极端随机树、CatBoost、梯度提升树和极端梯度提升树。在进行了详细的超参数调整和模型比较后,他们发现CatBoost模型提供了最准确的预测结果。
Schuetter及同事对不同的预测模型进行了评估,包括简单线性回归、随机森林、支持向量机、梯度提升树和多维克里金方法,用以预测油井的生产指标。他们还利用决策树分析了各个模型中特征的重要性,以区分高产和低产油井,并对这些变量的重要性进行了探究。
Meng及同事研究了页岩气吸附行为的预测,建立了人工神经网络、随机森林、支持向量机和极端梯度提升四种机器学习模型。他们将这些模型的预测效果与传统的基于压力和密度的等温模型以及压力温度统一模型进行了比较。研究结果显示,机器学习模型能够提供准确的预测,并克服了温度和岩石类型的限制,其中极端梯度提升模型表现最佳。
这段内容总结了多个研究项目,每个都专注于使用不同的机器学习技术来预测油气产量。Kubota团队提出了一个适应性强的线性模型来准确预测油井的产油率和产液率,强调机器学习可以补充传统油气工程工具。Li等人利用主成分分析和神经网络识别了影响页岩气产量的主要因素,并提出了提高产量的策略。Lin团队通过BP神经网络和数据分析方法,预测了生产能力,并确定了最优的生产参数。Han等人探讨了不同模型在预测累积产气量方面的效能,发现深度神经网络最有效。Amirian和同事们展示了使用Levenberg-Marquardt和反向传播方法构建的模型在预测准确度上的优势,并强调了超参数和数据量在模型训练中的重要性。Wang等人的研究通过深度神经网络模型分析了影响短期和长期产油量的关键因素,强调了超参数的重要性。
Chaikine和Gates使用混合卷积递归神经网络(C-RNN)来预测多级压裂井的五年累积产量。他们发现这种模型在预测单个油井产量时不仅准确度高,而且当整合多个井的数据时,预测的准确性会显著提高。
Panja及其团队通过使用人工神经网络(ANN)、最小二乘支持向量机(LSSVM)和二阶多项式响应面法(RSM)来建立一个考虑时间和速度限制的油井采收率预测模型。他们的研究结果显示,RSM和LSSVM在预测原油采收率方面比ANN更有效,且LSSVM在预测油气比方面展现出更高的准确性。
以上研究皆表明机器学习方法可以在油气产量预测上有很好的应用且效果优于传统方式。
\section{论文研究内容}
结合油田企业数字化转型的问题背景分析以及相关国内外研究现状,本文基于时间序列预测领域相关的技术方案设计了基于机器学习的气井智能管控系统。该系统的智能分析模块包括气井分类、产气量预测和开关井推荐算法。本文的主要内容如下:
(1)给出了一种基于改进的K-Shape时间序列聚类算法的气井分类算法,用于解决气井分类的问题。首先,本文描述了气井产气的原理,分析了对不同气井进行分类管控必要性。然后对取得的气井数据进行数据清洗,建立了气井时间序列聚类
的模型。最终实验结果显示,该方法可以有效地对气井进行分类。
(2)为了帮助企业有效规划气井的开发和生产过程,本文分别针对不同的气井类型进行产气量预测,给出了一种基于transformer的气井产量预测算法。首先使用滑动窗口的方法对第三章已经清洗好的数据进行数据预处理,然后根据不同的协变量
对输入的历史时刻和未来时刻的特征经过一系列变换,得到一个抽象的表征,
然后分别输入到 Encoder 和 Decoder 中。Encoder 部分使用 GRU 网络来编码历史时刻
的信息,输出一个固定长度的向量,表示整个历史序列的含义。Decoder 部分使用自
注意力机制来解码未来时刻的预测值,每个预测值都是根据之前所有时刻的加权结
果得到的。通过该算法,系统可以预测出未来时刻气井的产气量。根据产气量预测的结果,本文采用果蝇优化算法来对气井的开关井策略进行推荐。
(3)设计了一个气井智能管控系统,依据该企业各种数据杂乱无章的痛点,使用Apache Hudi对煤
企业的数据进行采集和存储,使用ClickHouse提供强大的数据分析能力,搭建一个
数字化平台系统,帮助企业进行数据的存储、管理和分析。支持用户在系统中进行气井的智能分析,并支持数据的可视化展示,根据油田部门的部门体系设计了用户和权限管理功能。
\section{论文组织结构}
根据课题内容,论文主要组织结构如图\ref{fig:paperStructure}所示。
\begin{figure}[H]
    \centering
    \caption{论文主要章节结构图}
    \label{fig:paperStructure}
\end{figure}
根据图\ref{fig:paperStructure}的主要章节结构安排,本文主要由六个章节组成,各章节组织结构如下:

第一章:绪论。本章主要阐述了基于机器学习的气井智能管控系统的研究背景及研究意义,并对国内外当前的时间序列聚类、时间序列预测和机器学习在油气产量的应用现状进行了描述,最后简要阐述了论文的主要研究内容和论文的组织结构。

第二章:基础理论和相关技术。本章详细介绍了论文涉及的相关理论和相关技术,其中包括使用的算法技术、算法模型、数据库技术等。

第三章:基于改进的 K-Shape 时间序列聚类算法的气井分类算
法。本章为了识别具有相似生产行为和特性的气井群体,了解不同气井的行为模式,优化生产策略,实现更有效的资源管理和分配,提出了一个基于改进的 K-Shape 时间序列聚类算法的气井分类算
法。从算法流程和思路对算法模型进行介绍,对企业内部的数据进行清洗并验证,最终验证了模型的有效性。同时,通过第四章,先对气井分类后再做产气量预测,可以提高预测的准确性,它允许模型更好地理解和适应每个聚类内序列的特定特征和动态变化。

第四章:基于transformer的气井产量预测算法。