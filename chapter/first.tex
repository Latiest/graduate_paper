\chapter{绪论}
本章主要阐述了气井智能管控系统的研究背景及意义,对国内外油气数字化现状进行了调研。进一步解释了对气井进行分类、产量预测和开关井推荐的必要性。最后对论文主要研究内容和组织结构进行了简要说明。
\section{研究背景和意义}
天然气是世界上广泛使用的能源。2022年,世界天然气消费量3.94万亿立方米,全球油气勘探开发投资支出4934亿美元,其中我国全国天然气
消费量3646亿立方米,进口天然气1503亿立方米\cite{chinaGasGOv}。在我国的能源资源中有举足轻重的地位。

随着计算机技术的快速发展和软件技术的高速迭代,大数据、云计算、人工智能
等领域快速兴起,已成为当今计算机领域内技术方向上的中流砥柱。随着上述领域的
不断发展,利用计算机技术来应用于传统行业已经十分有必要。
尽管我国原材料工业数字化转型不断走向纵深,但仍面临计算机技术应用于产业不够深入等问题\cite{workplanForPetrochemical}。油气行业面对气井智能分析的需求以及现有数字化软件中的一系列问题,迫切的需要利用计算机技术进
行数字化的建设和发展。本文将使用计算机方法来对气井进行分类、对气井产气量预测、根据预测结果进行开关井推荐,并以此为基础设计一个包含数据管理和智能分析的气井智能管控系统。

目前,已经有相当多的学者开展了气井分类方
面的研究,气井分类也由静态参数(如井深、地理位置、储层参数等\cite{NXSH200603007})分类、动态参数(如累产气、递减率、采收率等\cite{FCYQ202102014})分
类,发展到了动静参数结合的多指标综合分类。但当前的气井分类仍然存在以下问题:(1) 现有方法
多是从产能评价角度进行气井分类,未能够充分考
虑气井生产状况,从气井生产管理角度开展气井分
类;(2) 为全面反映气井特征,在分类分析中纳入了
气井的储层参数、分析测试数据,如孔渗情况、无阻
流量等来反映气井的供气能力,这些数据在气井生
产过程中难以实时获取,时效性差,从而在很大程度
上影响分类效果;(3) 绝大多数的分类方法需要基于
现场管理认识,给出指定的分类界限,对管理经验要
求较高的同时,还具有相当的主观性\cite{SYZC202104015}。同时,对于苏里格气田而言,其属于河流相沉积的岩性致密气田,也是典型的三低(低渗透、低压力、低丰度)气田,不仅单井产量低,产量递减快,压力下降快,且积液严重,大多数为间歇性气井\cite{KTSY202306014}。

气井开采过程中,通常会产生一定量的液体,包括水和油。产液量的增加会影响气井的产气量,因为液体在井筒中的积累会增加井筒的阻力,使气流的速度降低,从而影响气井的产气量。具体而言,有以下几种情况:如果气井处于高压高产阶段,开关井对产液量的影响不大,因为气相能够提供足够的能量将液体带出井口。如果气井处于低压低产阶段,开关井对产液量有较大影响,因为气相不能有效地携带液体,导致积液现象。此时,如采用间歇生产方式(即周期性地开关井),可以利用关闭期间积累的压力差,在开启时将积聚在底部或中部的液体迅速排出。如果气井处于中等压力阶段,开关井对产液量的影响取决于具体情况。一方面,如果开关频率过高或关闭时间过短,则可能造成反复积排液、增加摩阻损失、降低生产效率;另一方面,如果开关频率过低或关闭时间过长,则可能导致底部水锥突破、增加水含率、降低天然气质量。
\begin{figure}[H]
    \centering
    \caption{气井示意图}
\end{figure}
对于这种既包括需要周期性开关的间歇井又有连续井的气田而言,不同井的特征随时间变化的差异很大,相对于基于传统特征的分类方法,采用时间序列聚类分析方法可以深入分析气井随时间变化的产气量,允许用户精准地识别出不同生产条件下气井的行为模式。时间序列聚类能够基于实际的生产动态,而不仅仅是静态和动态特征,提供更个性化、适应性更强的气井预测算法,从而可以根据预测结果来提供开关井策略。这有助于最大化生产效率,减少液体积累带来的问题,优化生产过程,从而在不同的生产阶段维持气井的最佳运行状态,提高整体的生产效能。

气井产能预测是气田开采过程中最基本的任务之一,是气井工作制度优化和治理的重要依据之一。常用的传统的预测气井产量的方式有两种:预测气井的绝对无阻流量和通过递减曲线分析来预测产量。气井的无阻流量指的是在气井井底流压降低到0.1MPa时,气井的产气量。实测无
阻流量指的是将气井彻底敞开,当井底流动压力和大气压保持相同时,测量产气量, 此即为实测无阻流量。 该方法被认为是不合理的 , 原因在于气管尺寸会影响到结果的准确
性。而且操作过程中形成的气体会有严重的消耗,在水的锥进以及砂子颗粒的磨损作用下气井会受到影响。递减曲线分析方法依赖于简化假设和历史生产数据,这可能不完全反映复杂的地质和工程条件,特别是在非常规气藏中。此外,该方法忽略了井筒和地层之间的非均质性,需要长期的生产历史来形成准确的分析,且未考虑经济因素,从而可能导致对未来产量的预测不够准确或适用。

鉴于传统预测方法有诸多弊端,很多学者采用机器学习如时间序列预测等方法来预测气井产量,目前油气产量时间序列预测的研究大多是同步时间序列预测,问题描述为:已知属性(Source)和预测目标(Target)在每一步相对应,且已知属性时间序列长度和预测目标时间序列长度相等时计算产量,例如基于循环神经网络适用油藏的产量曲线预测同期的月产油、产水、产气量研究。同步时间序列预测实际上是对未测量数据的计算,因此,能够应用的场景有限,例如:不能使用井的静态参数及已知生产曲线预测未来生产曲线,也不能结合训练好的模型来优化生产设计。
本文通过将静态变量,现在已知的未来变量和过去的动态变量融合到同一个网络中,可以进行多时间尺度的预测。

在对气井进行预测后,企业需要根据预测结果来进行开关井推荐。气井的产量通常受到管网两端压力差的限制。当管网两端的压力差较大时,气井的产量会增加;而当压力差较小时,气井的产量会减少。相应地,开关井操作会改变管网两端的压力差。一般情况下,开井会导致管网两端的压力差减小,而关井则会导致压力差增大。此外,当管网两端的压力差发生变化时,会对气井的稳定性产生影响。特别是在压力差较小时,气井的稳定性容易受到影响。因此,在开关井控制中,需要综合考虑管网两端的压力差和气井的稳定性因素,制定最优的开关井策略,以确保气井的生产稳定性。
本文采用果蝇优化算法,根据用户期望产量,基于管网模型计算理论值与计算值的差值,来作为开关井选择的标准。

同时,企业不同部门的数据杂乱无章,数据格式不统一,内容不共享,用户往往需要维护多个平台的账号信息,学习每个平台的使用方式,使用Excel等格式文件在各平台间导入导出数据,导致管理十分混乱。随着企业战略发生变化,需要高度整合自身资源,避免数据分析

\section{国内外研究现状}