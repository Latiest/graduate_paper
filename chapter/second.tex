\chapter{相关技术概述}
\section{相关算法}
\subsection{RFM模型}
RFM分析模型\cite{hughes2005strategic}是一种基于客户行为数据的分析方法。该模型站在企业的视角,通过综合考量客户的一般购买行为,帮助企业评估和理解客户的价值。RFM模型主要依据三个关键维度来评价客户价值:近度(Recency, R)、频度(Frequency, F)和额度(Monetary, M)。

近度(Recency, R):此维度衡量的是自客户最后一次消费行为至样本采集时间点的间隔长度。这个间隔时间越短,表示客户最近对产品或服务的兴趣越高,相应地,这类客户的价值就更高。在具体的业务场景中,近度指标需要根据企业的特点和市场环境加以调整和解释。

频度(Frequency, F):该维度反映的是在特定时间段内客户消费行为的总次数。频率更高的客户表明他们与企业的交互更为频繁,通常意味着更高的客户忠诚度和更大的价值。企业需要基于历史交易数据来分析客户的消费频次,以识别出忠实用户群体,并制定策略以增强客户粘性和防止客户流失。

额度(Monetary, M):此指标是指在一个确定的时间段内客户的总消费额。一般而言,消费额度越高的客户被认为对企业更为重要,因为他们为企业带来了更多的收入。基于用户的累计消费额,企业可以识别出高价值客户,并通过制定VIP机制或者其他形式的客户回馈机制来提升这些客户的体验和满意度。

通过对近度、频度和额度三个维度的综合考量,企业能够更精准地理解和满足客户的需求,从而促进客户忠诚度的提升和企业收益的增长。
\subsection{核密度估计}
核密度估计\cite{10.1214/aoms/1177728190}(Kernel Density Estimation,简称KDE)是一种用于估算随机变量的概率密度函数的非参数方法。相对于传统的直方图估计,核密度估计能提供更为平滑的概率分布估计。该方法对于探索数据的潜在分布特征尤为重要,尤其是在缺乏关于数据分布先验知识的时候。

在核密度估计中,每个样本点都被视为一个“核”,这个核可以被理解为围绕样本点的一种平滑的概率质量分布。核函数通常会选择满足一定条件的函数,如非负性、积分为一(确保整体是概率分布)等,常见的核函数有高斯核、均匀核、三角核等。

核密度估计的公式如式\eqref{eq:KDE}所示。
\begin{equation}
    f(x) = \frac{1}{n}\sum_{i=1}^{n} \frac{1}{h}K\left(\frac{x-x_i}{h}\right)
    \label{eq:KDE}
\end{equation}
其中,\(f(x)\) 是估计的密度函数,\(n\) 是样本数量,\(x_i\) 是单个样本点,\(K(\cdot)\) 是核函数,\(h\) 是带宽参数,控制核的宽度,即平滑参数。带宽的选择对核密度估计的结果影响显著:带宽过小会导致估计结果过于崎岖,呈现过多的波动,即过拟合;而带宽过大则会导致估计过于平滑,无法准确反映数据的分布特性,即欠拟合。

在实际应用中,核密度估计广泛用于数据科学和统计学中,用于数据可视化、概率分布估计、异常点检测以及作为其他统计方法的基础。在进行核密度估计时,合理选择核函数和带宽参数对于获得准确、可靠的估计结果至关重要。通过核密度估计,研究者可以更加直观地理解数据集的分布特征,为后续的数据分析和决策提供支持。

\subsection{门控循环单元}
门控循环单元(Gated Recurrent Unit, GRU)是循环神经网络(Recurrent Neural
Network, RNN)的变体。循环神经网络(RNN)是一种专门用于处理序列数据的神经网络。它的内部结构包括一个循环单元,用于存储过去的信息,使网络能够利用之前的信息来处理当前的输入。这种结构允许信息随时间传递,使得RNN特别适用于时间序列数据或任何形式的有序数据。在RNN中,每个节点都接收前一个时间步的输出和当前时间步的输入作为其输入,创建了一个在时间上循环的网络。
原始的RNN在处理长期依赖问题时表现不佳,容易出现梯度消失和梯度爆炸的问题,这使得模型难以从长序列中有效地学习信息。为了解决这个问题,研究者们提出了LSTM(长短期记忆)和GRU(门控循环单元)两种变体,通过加入门控的方式选择性地保留长期依赖。
与传统的RNN相比,LSTM的独特之处在于其内部结构,它包含有三个重要的门控机制:忘记门、输入门和输出门,以及一个持久的细胞状态,这些共同工作以保持和调节信息流。其中忘记门负责决定哪些信息应从细胞状态中移除,这通过观察当前输入和上一个时间步的隐藏状态来决定,帮助网络忘记那些不再重要的信息。
输入门决定哪些新信息将被存储在细胞状态中,使网络能够更新其存储的信息以反映新接收到的重要数据。输出门控制从细胞状态到隐藏状态的信息流,确定下一个时间步的输出应该包含哪些内容。
细胞状态则像一条传送带,横贯整个LSTM结构,只有通过门控制的微小调整,允许信息无阻碍地流过整个链。这种设计使得LSTM能够在长时间间隔内保存状态和记忆,有效避免了传统RNN中常见的梯度消失问题。
GRU是一种与LSTM相似的RNN架构,但结构上更为简化。其旨在解决传统RNN无法有效处理长期依赖问题的同时,减少计算复杂度和模型参数量,从而加快训练速度并减少计算资源需求。
GRU的关键在于它将LSTM中的三个门(忘记门、输入门和输出门)简化为两个门:更新门(Update Grate)和重置门(Reset Grate)。其中更新门决定了从一个状态到另一个状态有多少信息是需要更新的。更新门帮助模型决定在当前状态的信息中保留多少旧信息和加入多少新信息。它类似于LSTM中的忘记门和输入门的结合。
重置门用来决定在计算新状态时应该忽略多少之前的信息。通过这种方式,重置门允许模型抛弃和忘记不重要的信息,有助于捕捉数据中的短期依赖特征。
除了门控机制,GRU中还有一个重要的概念是隐藏状态(Hidden State),它类似于LSTM中的细胞状态和隐藏状态的结合体,但更加简化。隐藏状态在每个时间步被更新,并用于捕获并传递序列中的信息。
与LSTM相比,GRU的优势在于模型结构更简单,这意味着它们需要更少的参数,从而减少了内存消耗,加快了训练速度。
\subsection{Transformer模型}
在神经网络发展过程中,面对计算性能的限制和模型复杂度的增加,信息过载成为了一个突出问题。为应对这一挑战,业界借鉴人脑处理信息的策略,提出了注意力机制(Attention)。这一机制通过集中关注对当前任务更为关键的信息,同时忽略不相关的数据,从而提高神经网络的处理效率和性能。

注意力机制的核心原理是对信息进行加权处理,增加关键信息的重量,减少无关信息的影响。这一机制通过三种向量实现:查询向量$\mathbf{Q}$(Query)、关键向量$\mathbf{K}$(Key)和值向量$\mathbf{V}$(Value)。其中,查询向量代表与当前任务相关的需求,关键向量代表数据的关键属性,两者相互作用生成一个注意力分布,进而决定值向量中哪些信息是当前步骤需要关注的。

Transformer模型是一种完全基于注意力机制的架构,由Google公司开发\cite{vaswani2023attention}。它能够进行并行计算,有效捕捉长距离的依赖关系。该模型采用了编码器-解码器(Encoder-Decoder)结构,整个流程包括数据输入、编码处理、解码处理和最终输出四个主要部分,能够处理各种序列到序列的任务。

(1)数据输入

序列数据首先通过输入嵌入(Input Embedding)转换为嵌入向量,以此表示数据中的每个元素。随后,为了让模型能够识别序列中每个元素的位置,会为每个嵌入向量增加位置编码PE(Positional Encoding)。这样可以保持序列的顺序信息。
PE的计算过程如公式\eqref{pos2i}和公式\eqref{posi}所示。
\begin{equation}
    PE(pos, 2i) = \sin\left(\frac{pos}{10000^{2i/d_{\text{model}}}}\right)
    \label{pos2i}
\end{equation}
\begin{equation}
    PE(pos, 2i+1) = \cos\left(\frac{pos}{10000^{2i/d_{\text{model}}}}\right)
    \label{posi}
\end{equation}
式中,\textit{pos} 代表序列中某一数据点在整个序列中的绝对位置,\(d_{\text{model}}\) 代表数据的嵌入向量维度,而 \textit{i} 表示向量中的具体维度位置。此外,\(2i\) 和 \(2i + 1\) 分别代表向量中的偶数和奇数位置维度。这种表示方法有助于在位置编码过程中区分不同位置的数据点。

(2)Encoder

在Transformer模型的编码器中,已处理的输入数据以向量形式通过多头注意力机制,随后结合残差连接和层归一化操作(统称为Add\&Norm),以及通过前馈神经网络进行进一步处理。这一系列步骤旨在增强模型对不同输入数据间关系的理解,并提高模型的学习能力及其对序列数据的处理效率。
其中注意力机制的计算过程如公式\eqref{tranatten}所示。
\begin{equation}
    \text{Attention}(Q,K,V) = \text{softmax}\left(\frac{QK^T}{\sqrt{d_k}}\right)V
    \label{tranatten}
\end{equation}
在该文中,\(\mathbf{Q}\)、\(\mathbf{K}\)和\(\mathbf{V}\)分别是输入数据\(\mathbf{X}\)与相应的权重矩阵\(\mathbf{W^Q}\)、\(\mathbf{W^K}\)和\(\mathbf{W^V}\)相乘后得到的向量。这里,\(d_k\)表示向量\(\mathbf{K}\)的维度。softmax归一化指数函数被应用于这些向量的乘积,以将它们映射到一个概率分布上,其值位于区间[0,1]之内,并且所有数值的和为1。
多头自注意力层(MHSA)通过将查询、键和值向量分割成h个较小的子向量组,每组进行独立的自注意力计算,然后将这些结果拼接起来。这种方法不仅减少了计算量,还允许模型在不同的表示子空间中捕捉信息,增强了模型理解全局特征互相关性的能力。
其表达式如式\eqref{trasmuti}所示。
\begin{equation}
    \begin{aligned}
        \text{MultiHead}(Q, K, V) = \text{Concat}(head_1, \ldots, head_h)W^Q \\
        \text{where } head_i = \text{Attention}(QW_i^Q, KW_i^K, VW_i^V)
        \label{trasmuti}
    \end{aligned}
\end{equation}
通过使用一组不同的权重矩阵\(W^Q\)、\(W^K\)和\(W^V\)来生成多组查询向量\(\mathbf{Q}\)、键向量\(\mathbf{K}\)和值向量\(\mathbf{V}\)。每一组的这些向量相互作用,分别产生一个输出矩阵\(\mathbf{Z}\)。最终,这些输出矩阵\(\mathbf{Z}\)被拼接成一个更大的矩阵,以获得综合的信息表达。   
Add\&Norm操作包括两个步骤:首先是残差连接,它将多头注意力的输出\(\mathbf{Z}\)矩阵与输入\(\mathbf{X}\)矩阵相加;其次是归一化,使用层归一化(Layer Normalization)技术对加和后的结果进行标准化,确保数据在一个合理的范围内。
前馈神经网络的计算过程如式\eqref{ffn}所示。
\begin{equation}
    \text{FFN}(x) = \max(0, xW_1 + b_1)W_2 + b_2
    \label{ffn}
\end{equation}
x代表从多头注意力机制得到的输出\(\mathbf{Z}\)矩阵。这个矩阵首先通过一个线性变换过程,然后应用ReLU激活函数进行非线性转换,筛选重要的信号。最后,这个过程通过另一个线性变换完成,产生最终的输出。

(3)Decoder 

Decoder和Encoder的计算原理基本相同,但Decoder增加了掩码多头注意力机制,这一机制通过引入掩码(mask)来修改多头注意力计算过程。其中,padding mask用于处理不同长度的序列,保证数据对齐,而sequence mask防止Decoder提前看到未来的信息,确保每个位置只能依赖于它之前的位置,从而保护序列数据的时序完整性。

(4)输出

从Decoder接收的输入数据首先经历一次线性变换,然后通过softmax函数进行处理,最终输出预测结果。

\section{数据存储}
\subsection{数据湖架构与 Apache Hudi}
为了克服数据管理领域中存在的特定缺陷,Dixon 引入了“数据湖”这一概念\cite{sawadogo2021data}。此概念应对了传统数据仓库存在的两大核心问题:首先,现行的元数据管理方案缺乏统一性,导致数据孤岛现象,妨碍了数据集之间的互通;其次,对原始数据缺乏适当存储,结果是重要的初始数据经常被遗失。
数据湖理念的提出,旨在通过提供一个集中式、灵活的数据存储解决方案,解决上述问题。

随后,数据湖的架构随着云计算技术的快速发展而演进。特别是在2015年,各大企业开始利用云存储的大规模、高可用性和低成本优势,逐步替代传统的HDFS存储方案。这一新型架构支持不同格式的数据存储,并采用分离计算与存储的架构,使其能够与多种计算引擎协同工作。

进入2019年,Hudi 数据格式\cite{Hudi}的推出进一步推动了数据湖技术的发展。Hudi 提供了一种特殊的存储格式,并引入了一种增量处理框架,它可以与 Apache Spark 等计算平台协同工作,并支持ACID事务以及增量更新\cite{gebretsadkan2019hudi},使得数据的更新和写入更加高效。

Hudi 的架构包括三个主要组件:索引(通过索引机制优化写入和更新速度,维护文件ID与Hoodie键的映射关系)、数据文件(存放实际数据)以及时序元数据(相似于数据库的事务日志)。Hudi 支持两种类型的数据表:写时复制(COW)表和读时合并(MOR)表\cite{belov2021analysis}。
COW表优化了读取操作,适合读取密集型应用;而MOR表优化了写入操作,适合写入密集型应用。

在数据查询方面,Hudi 支持快照查询、增量查询和读取优化查询。这些查询类型为用户提供了灵活的数据访问方法,允许他们根据需要选择最合适的查询模式。尽管MOR表在进行快照查询时可能会有延迟,但它们为写入密集的应用场景提供了优化方案。
\subsection{ClickHouse}
ClickHouse 是一种专为列式存储而设计的数据库管理系统,特别适用于实时分析场景。最初,它被应用于网络分析服务,后来转为开源项目。它的特点包括不存储元数据和最小化非必要数据,从而最大化数据处理的吞吐量,这使得ClickHouse非常适合于OLAP(在线分析处理)应用场景\cite{goranssonevaluating}。

与传统的行式数据库相比,在列式存储数据库中数据是按列序列化存储的,这样在查询或处理跨多行的大量数据时更加高效\cite{dwivedi2012performance},因为它只针对相关的数据列进行搜索和处理。而行式存储的数据库在添加数据时必须逐一定义每行的全部属性,导致在选择单行的多列时效率较高,但在大量数据处理方面则显得不足。

ClickHouse 的架构主要基于“列”和“字段”的概念。在这种基于列的存储结构中,一组数据被封装在一个“列”对象中,而“字段”则表示列中的某个特定值。在ClickHouse中,数据操作通常以列为单位进行,当需要处理特定的值时,则需使用“字段”对象。此外,“数据类型”用于执行来自“列”或“字段”的数据的序列化操作。而“数据块”是ClickHouse中进行数据操作的基本单位,包含上述三种对象。

在功能方面,ClickHouse 区分了“普通函数”和“聚合函数”。普通函数,通过 IFunction 接口实现,提供了许多实用函数,并以数据块为单位进行操作,而不是单独处理每行数据。与此相反,聚合函数则由 IAggregateFunction 接口定义,它们是有状态的,并且它们的状态可以进行序列化,因此能够在分布式节点之间进行增量计算。

在数据表设计中,ClickHouse 没有使用“表”对象,而是采用 IStorage 接口来代表数据表,该接口定义了数据定义、读取和写入的方法。这允许 IStorage 根据查询语句的抽象语法树需求,返回查询列的原始数据。

ClickHouse 服务器为多种客户端交互提供了接口,包括用于前后端交互的 HTTP 接口、用于本地客户端交互和执行分布式查询的 TCP 接口,以及用于复制传输信息的接口。服务器的作用是接收客户端提交的 SQL 语句并进行数据查询处理。其中,Parser 负责创建抽象语法树,Intercepter 负责解释抽象语法树,结合 IStorage 可以完成整个数据查询过程。因此,整个数据查询流程是在数据块上进行的,首先通过解析器将查询语句解析成抽象语法树,然后通过解释器对生成的抽象语法树进行解析,并创建查询通道。最终,IStorage 根据抽象语法树的指导返回查询列的数据。

\section{本章小节}
本章介绍了对气井分类和产气量预测所涉及到的相关算法,包括核密度估计、RFM模型、Transformer和GRU等。同时对于数据存储所需要的相关技术进行了介绍。