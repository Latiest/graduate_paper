\begin{abstract}
    当前,数字化转型的浪潮汹涌而至,石油天然气行业亦面临前所未有的变革与挑战。其中,作为气井行业的中流砥柱,某油田企业的气井数量逐渐增多,老井管理渐渐难度加大。在此背景下,该企业深知数字化建设之必要,然而其目前仍处于以传统经验主导生产管理与决策制定的阶段,
    这使其难以适应新时代的发展要求。
    此外,由于其早期对数据管理的忽视,导致了珍贵的数据资产散落于各个不同的系统,再加上大量数据都由其员工线下掌握,导致该企业数据孤岛问题日益突出,影响了其决策的时效性与准确性。

    基于以上痛点,受企业委托,本文提出了气井智能管控系统解决方案。该系统引入了机器学习、大数据分析等前沿技术,旨在实现气井全的智能化管理,进而提升气井生产管理与决策制定的质量和效果。
    
    本文的主要研究工作如下:
    
    (1)考虑到不同气井呈现出的不同的生产特性,本文创新性地将原本用于营销领域的RFM(Recency, Frequency, Monetary) 模型应用于气井分类问题。通过将RFM模型与核密度估计(Kernel Density Estimation,KDE)方法结合,提出了一种新颖的气井分类算法。
    与传统聚类算法和时间聚类算法相比,该方法结果更稳定,且具备清晰的物理意义,有助于气井分类管理。该分类方法在后续产气量预测中亦能显著提升预测效果。
    
    (2)过去企业在需要制定方案时,经常依赖经验盲目决策。因此,企业希望可以对气井产气量进行准确预测,以帮助他们使用数据驱动的方式科学地做出决策。基于此需求,本文提出一种基于Transformer的时间序列预测算法,将其用于气井的产气量预测。
    该算法可综合利用混合输入(包括静态信息、过去已知未来不可知信息及过去未来已知历史数据),采用门控循环单元(Gated Recurrent Unit, GRU)学习局部特征,利用自注意力机制学习不同时间步间的长期依赖关系。
    最终实验效果显示其可以较为准确地预测未来产气量,为企业优化生产管理与决策提供了科学依据。
    
    (3)针对企业软件应用分散、数据孤岛等问题,本文进行了详细的需求分析的基础上,设计了一个包含数据连接、数据管理和智能分析三大模块的气井智能管控系统。该系统可有效整合分散数据资源,消除信息孤岛。同时,其智能分析模块提供了气井分类、产气量预测、开关井推荐等功能,可以大幅提升数据利用率与决策质量。

\keywords{RFM 、核密度估计、时间序列预测、数据管理}
\end{abstract}
\begin{englishabstract}
    Currently, the wave of digital transformation is surging, and the oil and natural gas industry is also facing unprecedented changes and challenges. In particular, as the backbone of the gas well industry, a certain oilfield company has seen an increasing number of gas wells, making the management of old wells increasingly difficult. Against this backdrop, the company is fully aware of the necessity for digital construction; however, it is still at a stage dominated by traditional experience in production management and decision-making, which makes it difficult to meet the development requirements of the new era. Moreover, due to its early neglect in data management, valuable data assets are scattered across various different systems. Additionally, a large amount of data is held offline by its employees, leading to an increasingly prominent issue of data silos within the company, affecting the timeliness and accuracy of its decision-making.

    Based on the above pain points, commissioned by the enterprise, this paper proposes a gas well intelligent control system solution. This system introduces cutting-edge technologies such as machine learning and big data analysis, aiming to achieve fully intelligent management of gas wells, thereby improving the quality and effectiveness of gas well production management and decision-making.
    
    The main research work of this paper is as follows:
    
    (1) Considering the different production characteristics presented by different gas wells, this paper innovatively applies the RFM (Recency, Frequency, Monetary) model, originally used in the marketing field, to the gas well classification problem. By combining the RFM model with the Kernel Density Estimation (KDE) method, a novel gas well classification algorithm is proposed. Compared with traditional clustering algorithms and temporal clustering methods, this method yields more stable results and has a clear physical meaning, which helps in the management of gas well classification. This classification method can also significantly improve the prediction effect in subsequent gas production forecasting.
    
    (2) In the past, enterprises often relied on experience for blind decision-making when needing to formulate plans. Therefore, the enterprise hopes to accurately predict the gas production volume to help them make scientific decisions in a data-driven manner. Based on this need, this paper proposes a time series prediction algorithm based on the Transformer, which is applied to predict the gas production volume of gas wells. The algorithm can make comprehensive use of mixed inputs (including static information, past known future unknown information, and past and future known historical data), utilize Gated Recurrent Unit (GRU) to learn local features, and employ self-attention mechanisms to learn long-term dependencies between different time steps. The final experimental results show that it can predict future gas production volumes relatively accurately, providing a scientific basis for the enterprise to optimize production management and decision-making.
    
    (3) Addressing the issues of dispersed enterprise software applications and data islands, this paper, based on detailed requirement analysis, designs a gas well intelligent control system consisting of three major modules: data connection, data management, and intelligent analysis. This system can effectively integrate scattered data resources and eliminate information silos. At the same time, its intelligent analysis module provides functions such as gas well classification, gas production volume prediction, and well on/off recommendations, which can greatly improve data utilization and decision-making quality.
\englishkeywords{RFM (Recency, Frequency, Monetary) 、Kernel Density Estimation (KDE) 、Time Series Prediction 、Data Management}
\end{englishabstract}
\XDUpremainmatter
% \begin{symbollist}{p{13.5em}p{20.5em}X}
% % 符号 & 符号名称\\
% % $[N]$                              &关联矩阵\\
% % $[N]^+$                            &输出关联矩阵\\
% % $[N]^-$                            &输入关联矩阵\\
% % $\mathbb{N}$                       &自然数集合\\
% % $\mathbb{N}^{+}$                   &正整数集合\\
% % $N$                                &Petri网\\
% % $P$                                &库所集合\\
% % $p$                                &一个库所\\
% % $p^\bullet$                        &库所$p$的后置\\
% % $^\bullet p$                       &库所$p$的前置\\
% % $\mathbb{Q}_{0}^{+}$               &正有理数集合\\
% % $R(N, M_0)$                        &网$(N, M_0)$的状态空间\\
% % $RG(N, M_0)$                       &网$(N, M_0)$的可达图\\
% % $\mathbb{R}^{+}$                   &正实数集合\\
% % $T$                                &变迁集合\\
% % $t$                                &一个变迁\\
% % $t^\bullet$                        &变迁$t$的后置\\
% % $^\bullet t$                       &变迁$t$的前置\\
% 符号 & 符号名称 \\
% R & 最小开井近度 \\
% F & 开井频度 \\
% M & 平均开井产量 \\
% WellNo & 气井号\\
% Cluster & 井丛号\\
% Station & 集气站号\\
% $\delta $ & 井口压力\\
% $\rho $ & 套压\\
% $\sigma $ & 井口温度 \\
% $\varphi $ & 日产量 \\
% $\tau $ & 投产天数 \\
% $h$ & 每日开井时间 \\

% \end{symbollist}
\begin{abbreviationlist}{p{3cm}p{6.5cm}X}
% 缩略语 & 英文全称 & 中文对照\\
% TPN           &Time Petri Net                              &时间Petri网\\
% TdPN          &Timed Petri Net                             &时延Petri网\\
% TTPN          &Time Transition Petri Net                   &带时间变迁的Petri网\\
% TPPN          &Time Place Petri Net                        &带时间库所的Petri网\\
% TAPN          &Time Arc Petri Net                          &带时间弧的Petri网\\
% TTPPN         &Time Transition Place Petri Net             &变迁库所时间网\\
缩略语 & 英文全称 & 中文对照 \\
RFM  & Recency, Frequency, Monetary model & “最近一次购买时间、购买频率、购买金额”模型 \\
KDE  & Kernel Density Estimation & 核密度估计\\
MAE  & Mean Absolute Error & 平均绝对误差 \\
RCPE & Relative Change Percentage Error & 相对误差 \\
LightGBM & Light Gradient Boosting Machine & 轻量级梯度提升框架 \\
TFT & Temporal Fusion Transformers & 时间融合Transformer \\
SQL & Structured Query Language & 结构化查询语言\\
\end{abbreviationlist}
