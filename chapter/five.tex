\chapter{实验结果及应用}
本章将展示气井分类和产气量预测的实验结果,并根据未来七天的产气量预测的结果,结合企业专家提供的开关井经验,对对井的开关井策略进行推荐。
\section{数据集描述与环境说明}
算法会在会在第\label{cha:data}节提供的数据集上进行验证。实验的环境如表\ref{tab:predicEnHard}所示。
% 软件环境如表\ref{tab:preicEnSO}所示。
\begin{table}[H]
    \caption{产气量预测实验环境}
    \label{tab:predicEnHard}
    \begin{tblr}{hlines,vlines,
        columns = {valign=m,co=-1},
        rows    = {halign=c},
        cell{1}{1}={r=2}{c},
        cell{3}{1} ={r=2}{c}
        }
        硬件环境 & 处理器 & 显卡 & 内存 & 硬盘 \\
        & Inter(R) Xeon(R) Silver 4261 cpu @ 2.10Hz & GeForce RTX 3090 & 256G & 1T \\
        软件环境& 操作系统 & 开发环境 & Anaconda & Python \\
        & Windows 10 & Pycharm & 4.12.0 & 3.9 \\
    \end{tblr}
\end{table}
% \begin{table}
%     \caption{产气量预测实验软件环境}
%     \label{tab:preicEnSO}
%     \begin{tblr}{hlines,vlines,
%         columns = {valign=m,co=-1},
%         rows    = {halign=c},}
%         操作系统 & 开发环境 & Anaconda & Python \\
%         Windows 10 & Pycharm & 4.12.0 & 3.9 \\
%     \end{tblr}
% \end{table}
\section{气井分类实验结果}
由第三章可知,根据结合RFM与核密度的气井分类算法,气井将被分为四类,每一类的结果如表\ref{tab:everyclassnumber}所示。图\ref{fig:classResusca}为分类结果的展示。
\begin{figure}[H]
    \centering
    \subfloat[数值结果]{\includegraphics[width=.45\linewidth]{figure/RFM_count_pie.pdf}%
    \label{fig:wellclassbar}}
    \hfil
    \subfloat[RFM维度图]{\includegraphics[width=.45\linewidth]{figure/RFM_scatter.pdf}%
    \label{fig:wellclasssca}}
    \caption{气井分类结果(type1:活跃高价值 type2:不活跃高价值 type3:活跃一般价值 type4:不活跃一般价值)}
    \label{fig:classResusca}
\end{figure}
\begin{table}[H]
    \caption{气井分类结果展示表}
    \label{tab:everyclassnumber}
    \begin{tblr}{hlines,vlines,
        columns = {valign=m,co=-1},
        rows    = {halign=c},}
        气井类别 & 数量 \\
        活跃高价值 & 447 \\
        不活跃高价值 & 266 \\
        活跃一般价值 & 332 \\
        不活跃一般价值 & 24 \\
    \end{tblr}
\end{table}

\section{产气量预测实验结果}
\subsection{评估指标}
为了验证本章所提出的基于transformer的气井产量预测算法的有效性,使用两个指标用于评估。其中一个是平均绝对误差(Mean Absolute Error, MAE),其可以衡量
预测值与实际值之间的平均绝对偏差。公式如式\eqref{eq:MAE}所示。
\begin{equation}
    MAE(y, y') = \frac{1}{n} \sum_{i=1}^{n} |y_i - y'_i|
    \label{eq:MAE}
\end{equation}
另一个常用误差指标为累计产量的相对误差,公式如式\eqref{eq:RCPE}所示。
\begin{equation}
    RCPE = \frac{\sum_{i=1}^{n} |y_i| - \sum_{i=1}^{n} |y'_i|}{\sum_{i=1}^{n} |y_i|}
    \label{eq:RCPE}
\end{equation}
上面两式中,$i$表示索引,用于遍历数据集中的每一个数据点。例如,如果我们有一个包含多个观测值的数据集,$i$就用来表示这些观测值的序号,从第一个观测值
到第$n$个观测值。$y$表示真实值,也就是实际产量,$\hat{y}$表示预测值,也就是通过模型预测得到的产量。

\subsection{对比实验}
(1)分类预测与不分类预测对比

本次实验将气井分类之后再预测和不分类直接预测进行对比,其对比结果如表\ref{tab:classvalid}所示。为对比是否分类对预测结果的影响,使用本文模型针对未来7天的产气量构建预测模型,分别对每类型井的预测指标和不分类直接进行预测的结果进行统计,使用每类型气井预测结果的中位数作为对
该类预测效果的近似表示。
\begin{table}[H]
    \caption{分类预测与不分类预测结果对比}
    \label{tab:classvalid}
    \begin{tblr}{hlines,vlines,
        columns = {valign=m,co=-1},
        rows    = {halign=c},
        cell{1}{1} = {r=2}{c},
        cell{1}{2} = {c=2}{c},
        cell{1}{4} = {c=2}{c},
        cell{3}{2} = {r=4}{c},
        cell{3}{3} = {r=4}{c}
        }
        气井类别 & 不分类 & & 分类 & \\
          & MAE & RCPE & MAE & RCPE \\
        活跃高价值 & 0.477 & 0.058 & 0.225 & 0.030 \\
        不活跃高价值 & & & 0.367 & 0.051 \\
        活跃一般价值 & & & 0.103 & 0.042  \\
        不活跃一般价值 & & & 0.122 & 0.163  \\
    \end{tblr}
\end{table}
由表中可以看出,在不分类的情况下,预测结果的平均绝对误差(MAE)是0.477,相对百分比误差(RCPE)是0.058。除了不活跃一般价值的RCPE值高于不分类情况下的RCPE以外,这两个值在其他情况下均大于对每一类气井分别进行预测的结果。说明在进行气井产量预测时,先对气井进行分类再进行产气量预测在一定情况下时会达到比较好的效果的。

(2)消融实验

为了验证文中提出的基于Transformer模型中加入时间特征处理层和时域注意力层的有效性,本文对该算法设置消融实验来进行验证。将只加入时间特征处理层、只加入时域注意力层、两者都没有加的模型以及两者都加的模型(本文)在文中数据集上进行训练,
使用其针对未来七天的产气量进行预测,用结果指标的中位值来近视表示其预测结果。本文所得到的实验结果为对所有气井分类后,再按照气井的数量加权得到的。最终结果如表\ref{tab:ablation}所示。
\begin{table}[H]
    \caption{消融实验结果表}
    \label{tab:ablation}
    \begin{tblr}{hlines,vlines,
        columns = {valign=m,co=-1},
        rows    = {halign=c},}
        算法 & MAE & RCPE \\
        两者都无 & 0.736 & 0.377 \\
        仅无时域注意力层 & 0.436 & 0.113 \\
        仅无时间特征处理层 & 0.534 & 0.266 \\
        本文模型 & 0.220 & 0.042 \\
    \end{tblr}
\end{table}
由表中可知,将两个模块都去掉以后,模型的预测结果将显著下降,时间特征处理层对模型准确率的影响大于时域注意力层,这可能是由于气井产气量数据不具备明显的周期性,十分依赖过去的数据导致的。

(3) 对比模型选择

根据前文所述,本问题需要同时对1000余口气井构建时间序列预测模型,且问题包含多种类型的协变量,因此初步选取基于注意力机制的seq2seq模型,lightGBM模型、TFT模型以及本文的模型这四个可以解决这类问题的模型进行实验。

经过对手头的单井日报数据进行处理,累计10年,1000余口气井约有总共约有160多万条数据。为了展示各算法的预测性能,选取数量最多的一类井活跃高价值的产量数据作为测试集进行测试。如表\ref{fig:modelselection}所示,基于注意力机制的seq2seq模型,
由于模型结构限制,无法包含未来时刻的开关井信息,导致预测效果显著劣于其余三个模型。
lightGBM模型、TFT模型及本模型,预测准确率相近,本文算法略优于TFT和lightGBM模型,预测时间本文模型优于TFT和lightGBM模型。但是在模型训练阶段,lightGBM模型(10min)显著快于TFT模型(4h)及本文(2.5h)。故最终保留
本文中的模型和lightGBM模型。
\begin{table}[H]
    \caption{模型选择实验结果}
    \label{fig:modelselection}
    \begin{tblr}{hlines,vlines,
        columns = {valign=m,co=-1},
        rows    = {halign=c},
        cell{2}{1} = {r=4}{c}}
        测试对象 & 模型 & MAE($10^4 \ m^3$) & RCPE(\%)& 预测时间(s)\\ 
        活跃高价值气井& TFT          & 0.241                       & 3.14              & 8.369         \\       
                                    & LightGBM     & 0.262                       & 3.80              & 0.89     \\            
                                    & Seq2seq      & 0.512                       & 10.25             & 5.25   \\             
                                    & Ours         & 0.225                       & 3.02              & 7.29 \\               
    \end{tblr}
    % \centering
    % \begin{tabular}{|c|c|c|c|c|}
    %     \hline
    %     \textbf{测试对象} & \textbf{模型} & \textbf{MAE($10^4 \ m^3$)} & \textbf{RCPE(\%)} & \textbf{预测时间(s)} \\ \hline
    %     \multirow{4}{*}{活跃高价值气井}  & TFT          & 0.241                       & 3.14              & 8.369                \\ \cline{2-5}
    %                                 & LightGBM     & 0.262                       & 3.80              & 0.89                 \\ \cline{2-5}
    %                                 & Seq2seq      & 0.512                       & 10.25             & 5.25                 \\ \cline{2-5}
    %                                 & Ours         & 0.225                       & 3.02              & 7.29                 \\ \hline
    % \end{tabular}
\end{table}
结合图中可以看出,本文的模型在MAE和RCPE指标上的表现均优于其他三种模型,TFT由于参数量太复杂训练和预测时间都较久,效果也稍次于本模型。相对而言,LightGBM模型预测时间较短且也可以取得相对较好的预测结果。
这表明在实时性要求较高的场景中,LightGBM模型可能是一个较好的选择。Seq2seq模型在这三个测试对象上的表现相对较差,可能需要进一步优化模型结构或参数。根据实验结果,可以得出结论:在这三个测试对象上,本模型的预测性能最佳,
但预测时间较长。当实时性要求较高时,可以考虑使用LightGBM模型。

(4)使用部分数据进行预测

本次实验的动机为:由于早期数据缺乏套压记录,这可能会影响预测模型的准确性。近期数据包含了所有重要特征,包括套压数据,并且能够代表整个时间序列的主要趋势和模式,那么只使用近期数据进行预测可能是一个合理的选择。
\begin{table}[h]
    \caption{使用部分数据集训练和使用全数据集训练的对比}
    \label{tab:alldatacomparison}
    \begin{tblr}{hlines,vlines,
        columns = {valign=m,co=-1},
        rows    = {halign=c},
        cell{1}{1} = {r=2}{c},
        cell{1}{2} = {c=2}{},
        cell{1}{4} = {c=2}{},
        }
        气井类别 & MAE& & RCPE & \\ 
        & 使用部分数据 & 使用全部数据 & 使用部分数据 & 使用全部数据 \\ 
        活跃高价值               & 0.193        & 0.225       & 0.023        & 0.030        \\ 
        不活跃高价值               & 0.267        & 0.367        & 0.022        & 0.051        \\ 
        活跃一般价值               & 0.101        & 0.103        & 0.047        & 0.042        \\
        不活跃一般价值               & 0.120        & 0.122        & 0.165        & 0.163        \\ 
    \end{tblr}
\end{table}
为对比预测效果,使用本文模型针对未来7天的产气量构建预测模型,分别使用所有数据和使用2017-01-01之后的数据构建两个不同的模型。根据分类结果将模型划分为四类,使用上文选定的两种模型对这5类型的气井分别进行预测。

    
通过对每类型气井的预测指标进行统计,可以得出以下结论:

对于活跃高价值的气井,使
用2017-01-01之后的数据预测的MAE为0.193,而使用所有数据的MAE为0.225。其RCPE分别为0.028和0.030。这说明基于2017-01-01之后的数据构建的模型预测效果较好。对于不活跃高价值的气井,使用2017-01-01之后的数据预测的MAE为0.267,使用所有数据的MAE为0.367。其RCPE分别为0.022和0.051。这说明对不活跃高价值气井而言,基于2017-01-01之后的数据构建的模型预测效果较好。
对于活跃一般价值的气井,使用2017-01-01之后的数据预测的MAE为0.101,而使用所有数据的MAE为0.103。其RCPE分别为0.047和0.042。对这类气井而言,基于2017-01-01之后的数据构建的模型在MAE上略好,总体来看,这两个模型的预测效果相当。
对于不活跃一般价值的气井,使用2017-01-01之后的数据预测的MAE为0.120,而使用所有数据的MAE为0.122。同时,RCPE分别为0.165和0.163。对于这类气井而言,基于所有数据构建的模型在RCPE上略好,但综合两个指标来看,两个模
型的预测效果相当。

综上所述,在大部分气井类别中,基于2017-01-01之后的数据构建的模型预测效果都较好。

(5)本模型和LightGBM模型对比

实验基本设置: 在这个实验中,我们选用本模型和LightGBM对未来7天的产气量进行预测。实验结果是对属于该日均产气量范围内的所有气井进行预测并计算其指标的中位数统计值得到的。
\begin{table}[H]
    \renewcommand{\arraystretch}{1.5}
    \centering
    \caption{本文模型和LightGBM模型预测结果对比}
    \label{tab:prediction_comparison}
    \begin{tabular}{|c|c|c|c|}
    \hline
    模型     & 气井种类 & MAE($10^4 m^3$) & RCPE(\%) \\ \hline
    LightGBM & 活跃高价值       &0.234            & 0.042 \\ \hline
    LightGBM & 不活跃高价值     & 0.341           & 0.066    \\ \hline
    LightGBM & 活跃一般价值      & 0.136           & 0.093    \\ \hline
    LightGBM & 不活跃一般价值      & 0.128           & 0.228    \\ \hline
    Ours      & 活跃高价值        &0.225            &0.030      \\ \hline
    Ours      & 不活跃高价值      & 0.367           & 0.051   \\ \hline
    Ours      & 活跃一般价值      & 0.103           & 0.042    \\ \hline
    Ours      &不活跃一般价值      & 0.122           & 0.163    \\ \hline
    \end{tabular}
\end{table}

实验结果对比与分析: 根据上述统计结果表格,可以看出:

对于不活跃一般价值的气井,本文模型的MAE和RCPE分别为0.122和0.163,而LightGBM模型分别为0.128和0.228。对于这类气井而言,本文模型的预测性能更好。
对于活跃一般价值的气井,本文模型的MAE和RCPE分别为0.103和0.042,而LightGBM模型分别为0.136和0.093。在这种情况下,本文模型的预测性能也更好。
对于不活跃高价值的气井,本文模型的MAE和RCPE分别为0.367和0.051,而LightGBM模型分别为0.341和0.066。在这种情况下,尽管本文模型在RCPE上的表现更好,但在MAE上,LightGBM模型的表现更优。
对于活跃高价值的气井, 本文模型的MAE和RCPE分别为 0.225和0.030,而LiGBM模型分别为0.234和0.042。对于这类井而言,依然是本文模型的预测性能更好。

从预测性能上看,本文模型在各种情况下都表现出较好的预测性能。然而,从时间角度来看,LightGBM模型的训练和预测时间要远远低于本文模型。具体来说,LightGBM模型的重训练时间为5分钟,而本文模型为1小时;LightGBM模型
的预测时间平均约为1.2秒,而本文模型平均约为7.5秒。这意味着,在实际应用中,使用LightGBM模型会更加节省时间和计算资源。

总结:从实验结果来看,本文模型在预测性能上具有优势,但在时间方面,LightGBM模型具有更好的表现。因此,在实际应用中,需要根据实际需求的精度和计算资源进行权衡,选择合适的模型。
\subsection{实验结果统计值展示}
为完整展示实验结果,分别列出两种预测模型分别在预测步长为7天和45天时的预测结果统计指标,统计指标不仅包括中位数,还包括平均数,\%25分位数,75\%分位数,标准差。此处取所有分类结果的均值。预测步长7天的MAE结果如表\ref{tab:MAE7}所示。
\begin{table}[H]
    \renewcommand{\arraystretch}{1.5}
    \centering
    \caption{本文模型和LightGBM模型预测未来7天内产气量预测结果的MAE指标对比}
    \label{tab:MAE7}
    \begin{tabular}{|c|c|c|c|c|c|c|}
    \hline
    模型     & 气井类别 & MAE mean & MAE std & MAE 25\% & MAE 50\% & MAE 75\% \\ \hline
    LightGBM  &活跃高价值       &0.203     & 0.248   & 0.063    & 0.131     &0.244 \\ \hline 
    LightGBM & 不活跃高价值      & 0.279    & 0.366   & 0.087    & 0.163    & 0.297    \\ \hline
    LightGBM & 活跃一般价值     & 0.180    & 0.179   & 0.069    & 0.136    & 0.211    \\ \hline
    LightGBM & 不活跃一般价值      & 0.199    & 0.269   & 0.055    & 0.128    & 0.277    \\ \hline
    Ours      & 活跃高价值         & 0.189    &0.227    &0.059     & 0.134     &1.239  \\ \hline
    Ours       & 不活跃高价值      & 0.287    & 0.309   & 0.089    & 0.149    & 0.298    \\ \hline
    Ours      & 活跃一般价值     & 0.170    & 0.220   & 0.061    & 0.102    & 0.186    \\ \hline
    Ours     & 不活跃一般价值     & 0.174    & 0.236   & 0.053    & 0.122    & 0.205    \\ \hline
    \end{tabular}
\end{table}
预测步长7天的RCPE结果如表\ref{tab:RCPE7}所示。
\begin{table}[H]
    \renewcommand{\arraystretch}{1.5}
    \centering
    \caption{本文模型和LightGBM模型预测未来7天内产气量预测结果的RCPE指标对比}
    \label{tab:RCPE7}
    \begin{tabular}{|c|c|c|c|c|c|c|} % Change 'l' to 'c' for center alignment
    \hline
    模型     & 气井类别 & RCPE mean & RCPE std & RCPE 25\% & RCPE 50\% & RCPE 75\% \\ \hline
    LightGBM & 活跃高价值        &0.105     & 0.139     &0.029      &0.053       &0.113 \\ \hline
    LightGBM & 不活跃高价值      & 0.085     & 0.125    & 0.020     & 0.043     & 0.094     \\ \hline
    LightGBM & 活跃一般价值     & 0.131     & 0.147    & 0.038     & 0.093     & 0.167     \\ \hline
    LightGBM & 不活跃一般价值      & 0.433     & 0.845    & 0.097     & 0.228     & 0.477     \\ \hline
    Ours      &活跃高价值         &0.093     & 0.128     &0.022      & 0.031      &0.103 \\ \hline
    Ours      & 不活跃高价值      & 0.086     & 0.119    & 0.015     & 0.035     & 0.080     \\ \hline
    Ours      & 活跃一般价值     & 0.112     & 0.187    & 0.016     & 0.047     & 0.098     \\ \hline
    Ours      & 不活跃一般价值      & 0.363     & 0.757    & 0.066     & 0.163     & 0.358     \\ \hline
    \end{tabular}
\end{table}
预测步长45天的MAE结果如表\ref{tab:MAE45}所示。
\begin{table}[H]
    \renewcommand{\arraystretch}{1.5}
    \centering
    \caption{本文模型和LightGBM模型预测未来45天内产气量预测结果的MAE指标对比}
    \label{tab:MAE45}
    \begin{tabular}{|c|c|c|c|c|c|c|}
    \hline
    模型       & 气井类别 & MAE mean & MAE std & MAE 25\% & MAE 50\% & MAE 75\% \\ \hline
    LightGBM  & 活跃高价值        & 0.203   & 0.174    &0.098    & 0.178     & 0.298 \\ \hline
    LightGBM  & 不活跃高价值      & 0.360    & 0.329   & 0.151    & 0.253    & 0.418    \\ \hline
    LightGBM  & 活跃一般价值     & 0.245    & 0.191   & 0.106    & 0.186    & 0.306    \\ \hline
    LightGBM  & 不活跃一般价值      & 0.198    & 0.151   & 0.091    & 0.159    & 0.273    \\ \hline
    Ours       & 活跃高价值        &0.153     & 0.188   & 0.069     & 0.128   & 0.233   \\ \hline
    Ours       & 不活跃高价值      & 0.346    & 0.319   & 0.111    & 0.245    & 0.459    \\ \hline
    Ours       & 活跃一般价值     & 0.217    & 0.207   & 0.077    & 0.130    & 0.272    \\ \hline
    Ours       & 不活跃一般价值      & 0.132    & 0.113   & 0.058    & 0.110    & 0.181    \\ \hline
    \end{tabular}
\end{table}
预测步长45天的RCPE结果如表\ref{tab:RCPE45}所示。 
\begin{table}[H]
    \renewcommand{\arraystretch}{1.5}
    \centering
    \caption{本文模型和LightGBM模型预测未来45天内产气量预测结果的RCPE指标对比}
    \label{tab:RCPE45}
    \begin{tabular}{|c|c|c|c|c|c|c|}
    \hline
    模型       & 气井类别 & RCPE mean & RCPE std & RCPE 25\% & RCPE 50\% & RCPE 75\% \\ \hline
    LightGBM  & 活跃高价值       & 0.085      &0.097    &0.023      &0.051      & 0.104 \\ \hline
    LightGBM  & 不活跃高价值      & 0.091     & 0.107    & 0.028     & 0.056     & 0.114     \\ \hline
    LightGBM  & 活跃一般价值     & 0.117     & 0.121    & 0.029     & 0.074     & 0.160     \\ \hline
    LightGBM  & 不活跃一般价值      & 0.384     & 0.750    & 0.089     & 0.201     & 0.396     \\ \hline
    Ours       &活跃高价值          & 0.073    0.089      & 0.017     & 0.042     & 0.109 \\ \hline
    Ours       & 不活跃高价值      & 0.087     & 0.094    & 0.019     & 0.047     & 0.120     \\ \hline
    Ours       & 活跃一般价值     & 0.112     & 0.128    & 0.026     & 0.072     & 0.146     \\ \hline
    Ours       & 不活跃一般价值      & 0.361     & 0.731    & 0.071     & 0.146     & 0.319     \\ \hline
    \end{tabular}
\end{table}
为直观展示预测结果,选择6口气井的预测结果如\ref{fig:6predicre}所示。
\begin{figure}[H]
    \centering
    \subfloat[井XE8867-HU预测结果]{\includegraphics[width=.45\linewidth]{figure/forecast_SN0018-03.pdf}}
    \hfil
    \subfloat[井XE3878-RE预测结果]{\includegraphics[width=.45\linewidth]{figure/forecast_SN0037-01.pdf}}
    \hfil
    \subfloat[井SI2938-KI预测结果]{\includegraphics[width=.45\linewidth]{figure/forecast_SN0049-08.pdf}}
    \hfil
    \subfloat[井SI2938-KI预测结果]{\includegraphics[width=.45\linewidth]{figure/forecast_SN0075-04ST.pdf}}
    \hfil
    \subfloat[井LR7359-BR预测结果]{\includegraphics[width=.45\linewidth]{figure/forecast_SN0087-03i.pdf}}
    \hfil
    \subfloat[井RC8375-BN预测结果]{\includegraphics[width=.45\linewidth]{figure/forecast_SN0095-06.pdf}}
    \caption{气井预测结果展示}
    \label{fig:6predicre}
\end{figure}

\section{开关井推荐}
本小节将应用产气量预测的结果,根据企业日常开关井的规则进行开关井的策略推荐。
\subsection{算法设计}
气井开采过程中,通常会产生一定量的液体,包括水和油。产液量的增加会影响气井的产气量,因为液体在井筒中的积累会增加井筒的阻力,使气流的速度降低,
从而影响气井的产气量。当气井处于低压低产阶段时,气相不能有效地携带液体,会导致积液现象。此时,如采用间歇生产方式(即周期性地开关井),可以利用
关闭期间积累的压力差,在开启时将积聚在底部或中部的液体迅速排出。
\begin{algorithm}[H]
    \baselineskip=20pt
    \caption{开关井策略}
    \label{al:openclose}
    \begin{algorithmic}[1]
    \Require $df$: Dataset, $all\_forecasts$: 未来7天的预测结果, $auto\_well$: 自动切换井号, $target$: 目标值, $well\_no\_disable$: 不投入生产的井, $threshold$: 停产门槛, $pressure\_threshold$: 压力门槛,
    \Ensure $close\_wells$: 需要关闭的井列表, $open\_wells$: 需要开启的井列表, $close2open\_wells$: 从关闭状态切换到开启状态的井列表, $open2close\_wells$: 从开启状态切换到关闭状态的井列表
    
    \State 初始化 $close\_wells$, $open\_wells$, $close2open\_wells$, $open2close\_wells$
    \State 获取当前日期为 $dates\_time$
    \State 确定因压差过大需要临时关闭的井为 $need\_close\_by\_pressure$
    \State 获取当前开启和关闭井的信息为 $df\_open\_0$, $df\_close\_0$
    
    \For{$i = 1$ \textbf{至} $7$}
        \State 确定尚未达到开启时间的井为 $still\_need\_close\_no$
        \State 确定产量低于门槛需要关闭的井为 $need\_close\_by\_threshold$
        \State 确定在第$i$天需要关闭的井为 $need\_close\_no_i$ 
        \State 确定在第$i$天仍然可以开启的井为 $still\_can\_open\_no_i$ 
        \State 确定在第$i$天可以从关闭状态切换到开启状态的井为 $can\_close2open\_no_i$ 
        \State 确定在第$i$天可以从开启状态切换到关闭状态的井为 $can\_open2close\_no_i$ 
        \State 计算第$i$天的当前产量为 $cur\_production_i$
        \State 计算关闭井后的产量差值为 $delta_i$ 
        \If{$delta_i > 0$}
            \State 依次关闭开启时间较长的井来更新 $open2close\_wells_i$
        \Else
            \State 依次选择开启产量较高的井来更新 $close2open\_wells_i$
        \EndIf
        
        \State 更新第$i$天关闭的井列表为 $df\_close_i$
        \State 更新第$i$天开启的井列表为 $df\_open_i$
    \EndFor
    
    \State \Return $close\_wells$, $open\_wells$, $close2open\_wells$, $open2close\_wells$
    
    \end{algorithmic}
  \end{algorithm}

但如果开关频率过高或关闭时间过短,则可能造成反复积排液、增加摩阻损失、降低生产效率。可若是开关频率过低或关闭时间过长,则可能导致底部水锥突
破、增加水含率、降低天然气质量。

在此背景之下,本节基于第\ref{cha:data}节提供的数据,并借鉴专家经验,对开关井策略提出了建议。具体的推荐流程如算法\ref{al:openclose}所述。首先,进行产气量的预测,以估算第i天的产气量。然后,以企业设定的目标产气量$target$、不参与投入生产的井列表$well\_no\_disable$、根据经验规则自动进行开关控制的井列表$auto\_well$、以及设定的停产和压力门槛值$threshold$和$pressure\_threshold$为依据,进行开关井策略的制定。

策略制定的具体步骤包括:首先,排除因当日压力差异过大而暂时关闭的井,这些井在关闭后通常需要较长的时间进行调整。接着,获取当前的开关井状态信息。对于未来的第i天,需要排除那些根据企业规定尚未达到开井时间的井,并根据设定的停产门槛识别出产量低于阈值且需要关闭的井。

随后,综合考虑达到停产门槛、压力门槛的井、不投入生产的井以及企业规定第i天需保持关闭状态的井,来确定第i天所有需要关闭的井。其次,从已经关闭的井中,除了那些已经标记为不再投入生产的井之外,识别出尚未达到停产门槛和压力门槛的井,这些井是可以被重新开启的。

最后,计算当前产量,并在可开启和可关闭的井中进行选择,执行开关井操作。通过比较操作前后的产量差值,如果差值大于零,则应进一步关闭开启时间较长的井;如果差值小于零,则应优先开启产量较高的井。通过这一流程,最终确定开关井的结果。

\subsection{结果展示}
选取目标产气量为1000($10^4m^3$),压力阈值为2.0($MPa$),最终的得到的开关井推荐结果如图\ref{fig:openclosereco}所示。更详细的开井关井的具体信息将在第六章系统实现中展示。
\begin{figure}[H]
    \centering
    \includegraphics[width=.8\linewidth]{figure/reco.pdf}
    \caption{开关井推荐结果}
    \label{fig:openclosereco}
\end{figure}

  \section{本章小节}
  本章对前文提出的气井分类算法和气井产气量预测算法的结果进行了展示。气井被分为四类,通过对比实验发现,先分类再进行预测的准确率高于直接对气井进行预测的准确率。并通过消融实验证明了第四章基于Transformer的产气量预测算法模块的有效性。通过对实验结果树枝对比,发现GRU对模型准确性的
  影响大于Transformer。随后,本文通过实验选取了LightGBM作为本文算法的对比算法,经过实验得出本文算法在预测准确率上高于LightGBM算法,但在预测时间上慢于LightGBM算法,因此保留两种模型以供企业在需要进行产气量预测时根据不同的需要挑选模型。最后,本文应用了对产气量预测后的预测结果,
  根据企业的专家知识结合预测结果提出了开关井策略推荐算法,展示了算法流程,以及最后的推荐结果。