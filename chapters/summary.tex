\chapter{总结与展望}
\section{总结}
晶圆制造系统是用于生产半导体芯片的设备和工艺的集合体。
它由多个工艺步骤组成,包括晶圆清洗、切割、涂覆、曝光、蚀刻、离子注入、金属沉积等。
晶圆制造系统通常由多个设备组成,包括晶圆清洗机、曝光机、蚀刻机、离子注入机、金属沉积机等。
这些设备通常由自动化系统控制,以确保工艺步骤的准确性和一致性。
本文使用Petri网对实际的晶圆制造系统进行建模,并设计蚁群算法求此系统进行调度。

本文第二章介绍了Petri网的基本理论,总结了各种调度算法。
并且梳理了一系列时间Petri网的相关知识。

第三章根据晶圆制造系统的特点,结合了第二章介绍的变迁时间网可库所时间网,提出了一种新的时间网子类,变迁库所时间网。
并且设计了一种此时间网变迁发射的时间计算流程,以保证变迁发射后系统时间尽可能短,以及一个用于计算各种Petri网模型的程序架构,供后续调度算法使用。
之后使用变迁库所时间网对一个实际的晶圆制造系统进行建模。

第四章使用蚁群算法对第三章建立的模型进行调度。
但发现调度结果不能收敛到最优解,通过对解的分析,从不同角度提出了6种优化思路,
最终回溯蚁群算法效果显著。
\section{展望}
\subsection{变迁库所时间网的进一步优化思路}
如果晶圆需要进入一系列的加工腔加工,并且规定此晶圆从进入第一个加工腔开始,必须在规定时间内完成加工,最直观的做法是有某套机制能够跟踪晶圆。
晶圆由托肯表示,各托肯在Petri网中是无法区分的。
因此之后可以设计并开发一套托肯编号机制。

实际的Petri网系统中,既有托肯表示晶圆,也有表示逻辑值,这类托肯在被变迁移动时所进入的库所是不同的,
并且实际系统中存在多个晶圆交换操作。
因此应该设计一套托肯移动的规则,以保证上述逻辑的正确。
\subsection{蚁群算法的进一步优化思路}
对于蚁群算法的进一步优化,可以从时间和空间两个方面入手。
减少算法的时间开销可以通过阻止蚂蚁探索无意义的标识实现;
减少空间开销可以设计更高效的数据结构来实现。
\subsubsection{全局时间禁忌表}
本文结合Petri网的实际情况,提出了回溯蚁群算法并取得了不错的成果。
蚂蚁能够回溯后便拥有了发现不可行解的能力。
如果将这批不可行解记录下来,下一轮蚁群循迹时提前禁止蚂蚁走上不可行解,将大大提高蚁群算法收敛速度。
不可行解由标识组成,因此可以使用一个新的集合存储这些标识。
此集合用于避免蚂蚁走上无意义的标识,功能与禁忌表相似,
因此将其命名为全局禁忌表。

而本算法是基于时间Petri网的,
此Petri网超过时间约束也会引发死锁,
此类死锁受时间因素影响不能简单的判定为不可行解。

综上全局禁忌表存储的应该为一系列的键值对,键为不可行解中的标识向量,
值应为一系列的充分条件。
如果蚂蚁探索到的标识满足这一系列的条件,则意味着继续探索不可能到达终点标识。
这一系列的充分条件会随着蚁群的探索一步步细化,逼近充要条件。
\subsubsection{实现禁忌表新的数据结构}
如果将Petri网的标识向量编码为二进制串,有一种比哈希表更高效的数据结构,二元决策图可以以更低的空间开销存储二进制信息。
蚁群算法为多线程算法,更为适合在GPU上运行,GPU的显存带宽远高于内存带宽,但通常来说显存大小不如内存,并且显存与内存通信需要消耗时间。
因此可以使用布隆过滤器在显存和内存间做一层中间层,把显存中长期未处理过的标识转移入内存中。
内存与硬盘间也可以使用相似的逻辑,这样可以极大提升存储标识的数量。