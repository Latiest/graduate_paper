\begin{resume}
\section*{1.\hspace{0.75em}基本情况}
\anonrvwinfo{邱运义}{XXX},女,陕西安康人,1998年7月出生,西安电子科技大学\anonrvwinfo{计算机科学与技术}{XXX}学院\anonrvwinfo{软件工程}{XXX}专业2021级硕士研究生。
\section*{2.\hspace{0.75em}教育背景}
\begin{resumelist*}
\resumelistitem 2017.09~2021.06,西安电子科技大学,本科,专业:\anonrvwinfo{软件工程}{XXX}
\resumelistitem 2021.09~\phantom{},西安电子科技大学,硕士研究生,专业:\anonrvwinfo{软件工程}{XXX}
\end{resumelist*}
\section*{3.\hspace{0.75em}攻读硕士学位期间的研究成果}
\begin{resumelist}{\hspace{-0.25em}3.1\hspace{0.5em} 发表学术论文}
\resumelistitem \anonrvwinfo{XXX, XXX, XXX}{第一作者}. Rapid development technique for drip irrigation emitters[J].RP Journal,UK.,2003,9(2): 104-110.(SCI: 672CZ, EI: 03187452127)
\end{resumelist}
\begin{resumelist}{\hspace{-0.25em}3.2\hspace{0.5em} 申请(授权)专利}
\resumelistitem \anonrvwinfo{XXX, XXX, XXX等}{第一发明人}. 专利名称: 国别,专利号[P]. 出版日期.
\end{resumelist}
\begin{resumelist}{\hspace{-0.25em}3.3\hspace{0.5em} 参与科研项目及获奖}
\resumelistitem 实验室研发项目,《基于大数据分析及机器学习的气井全生命周期管理先导性试验项目》, 2022.8-2023.6, 项目交付完
成。该项目与某油气公司合作,使用深度学习中的时间序列预测模型预测每
口井的产量,并基于智能算法对开关井策略做出推荐。本人在该项目中负责
算法研究与实现和实验设计与实现等任务。
\resumelistitem 实验室研发项目,《面向涉钢行业的机器学习模型治理方法与技术的研究与应用》,2022.2-2023.6,项目交付完成。该项目与某钢铁研究所合作,为对涉钢产业中沉淀的机器学习模型进行综合治理,实现对模型的管理、部署和监控,提高机器学习模型开发的速度和质量,
突破钢铁行业在机器学习模型治理方面的空白。 本人在该项目中负责后端开发的工作。
\end{resumelist}
\end{resume}
